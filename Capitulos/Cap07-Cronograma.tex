\chapter{Cronograma}  \label{cap:07}

% Esse capítulo só se insere na Versão da Pré Banca. Na Versão da Defesa ele é retirado já que o cronograma deverá ter sido concluído.

\begin{table}[!h]
\begin{tabular}{|c|c|c|c|c|c|c|c|c|c|c|}
\hline
\multirow{2}{*}{\textbf{Descrição das Atividades}} & \multicolumn{10}{c|}{\textbf{Meses}}            \\ \cline{2-11} 
                                                   & 01 & 02 & 03 & 04 & 05 & 06 & 07 & 08 & 09 & 10 \\ \hline
Atividade 1                                        &    & $\bullet$ & $\bullet$  &    &    &    &    &    &    &    \\ \hline
Atividade 2                                        &    &    & $\bullet$  & $\bullet$  &    &    &    &    &    &    \\ \hline
Atividade 3                                        &    &    &    & $\bullet$  &    &    &    &    &    &    \\ \hline
Atividade 4                                        &    &    &    &    &    &    &    &    &    &    \\ \hline
Atividade 5                                        &    &    &    &    &    &    &    &    &    &    \\ \hline
Atividade 6                                        &    &    &    &    &    &    &    &    &    &    \\ \hline
\end{tabular}
\end{table}


No cronograma você irá inserir as atividades que irá realizar, bem como marcar com a bolinha nos meses que irá realizá-las. Como estamos no Latex, a edição dessas tabelas deve ser realizada em um programa que trabalhe com configuração de tabelas.
Para isso, você deve seguir os seguintes passos:

\begin{enumerate}
    \item Copie o código da tabela acima.
    \item Acesse o seguinte site: \href{https://www.tablesgenerator.com/latex_tables#}{Editor de Tabelas} ;
    \item Clique em "File";
    \item Clique em "From Latex Code";
    \item Cole o Texto da tabela copiada e clique em "Load";
\end{enumerate}

Esse procedimento irá gerar uma tabela como a que está acima. Basta editar conforme sua necessidade, alterando as atividades e o nome dos meses e posicionando a bolinha nos quadrados dos meses em que as atividades serão realizadas.

Posteriormente basta clicar em "Generate", copiar o código e substituir a tabela acima por ele.