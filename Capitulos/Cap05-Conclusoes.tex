\chapter{Conclusões} \label{cap:05}

A implementação de um aplicativo capaz de descrever o ambiente e apresentar as informações por meio de áudio para pessoas com deficiência visual teve, como ponto-chave, a comparação de diferentes LLMs. Essa comparação foi essencial para determinar o equilíbrio adequado entre qualidade das descrições e desempenho em tempo real, culminando na escolha do \texttt{Qwen2.5-VL-7B-Instruct} como a alternativa mais viável. Ao longo do trabalho, constatou-se que a menor latência e a robustez na geração de descrições deste modelo se destacaram em relação às demais opções avaliadas, contribuindo diretamente para uma experiência de uso mais fluida.

Os resultados obtidos demonstram que os objetivos foram alcançados: o desenvolvimento de um protótipo funcional, acessível e capaz de auxiliar usuários com deficiência visual na percepção do mundo ao redor. A utilização de LLMs \textit{open source} reforça a relevância social e tecnológica da solução, pois permite a evolução contínua do projeto com base em avanços na comunidade de pesquisa e desenvolvimento. Além disso, a adoção de uma arquitetura modular, na qual o aplicativo móvel e a API de processamento são facilmente intercambiáveis, reforça a flexibilidade do sistema, mostrando que modelos multimodais e avançados podem ser integrados de maneira relativamente simples.

Para trabalhos futuros, algumas direções podem levar a ganhos significativos de desempenho e acessibilidade. Em primeiro lugar, a execução direta dos modelos de LLM no dispositivo móvel, reduzindo ou eliminando a necessidade de conexão à internet, representa um passo importante para usuários em regiões de acesso limitado. Associada a isso, a aplicação de técnicas mais avançadas de otimização, seja pela redução de parâmetros ou pelo desenvolvimento de arquiteturas mais enxutas, pode diminuir a latência e os custos de execução. Também se vislumbra a possibilidade de aprimorar o TTS, tornando-o mais natural, e de estender a robustez do sistema, de modo que ele mantenha qualidade em diferentes condições ambientais, tais como iluminação adversa ou ambientes muito complexos. Dessa forma, o protótipo abre caminho para novas pesquisas e aprimoramentos, reforçando o papel das tecnologias assistivas no fortalecimento da inclusão digital.