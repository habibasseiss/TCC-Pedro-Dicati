\chapter{Objetivos}  \label{cap:02}

% Para a Banca Final, lembre-se de conversar com o orientador sobre suprimir esse capítulo já que ele terá sido cumprido, e inserí-lo na Introdução em modo Parágrafo. 
% Não apague esse comentário.


\section{Objetivo Geral}

O objetivo geral do trabalho é o elemento que resume e apresenta a ideia central do trabalho acadêmico. Normalmente é redigido em uma frase, utilizando o verbo no infinitivo. 

% textbf é um comando que deixa o texto em negrito
\textbf{Exemplo:} Detalhar o presente modelo de escrita para a produção do Trabalho de Conclusão de Curso.\\

\section{Objetivos Específicos}

% textit é um comando que deixa o texto em itálico
Os objetivos específicos definem os diferentes pontos a serem abordados, visando confirmar as hipóteses e concretizar o objetivo geral. Em suma, são as ações que serão desenvolvidas a fim de que se alcance o \textit{objetivo geral}. Devem ser escritos no infinitivo.

\textbf{Exemplo:}

\begin{itemize}
    \item Realizar a revisão bibliográfica sobre TCC;
    \item Criar um Tutorial sobre a utilização do Overleaf;
    \item Aplicar os conhecimentos adquiridos para edição de documentos.
\end{itemize}