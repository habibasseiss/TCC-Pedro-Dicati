\begin{resumo}

%\noindent{SILVA, João da. \textbf{Título do trabalho de conclusão de curso em negrito}. Nº de fls. TCC (Trabalho de Conclusão de Curso). Instituto Federal de Mato Grosso do Sul – IFMS. Tecnologia em Análise e Desenvolvimento de Sistemas, Câmpus Nova Andradina, MS. 2018.}

%\setlength{\absparsep}{18pt} % ajusta o espaçamento dos parágrafos do resumo
%\vspace{1.5cm}
	
A acessibilidade digital é um desafio crucial em um mundo cada vez mais conectado, especialmente para pessoas com deficiência visual. Este trabalho apresenta o desenvolvimento de um aplicativo multiplataforma assistivo capaz de capturar imagens do ambiente e descrevê-las em áudio, utilizando modelos de linguagem de grande porte (LLMs) de código aberto. O aplicativo integra recursos de visão computacional e síntese de voz (Text-to-Speech - TTS), promovendo maior autonomia para usuários com deficiência visual. A metodologia incluiu o benchmark de três modelos de IA (Qwen 2.5, llava v1.6 Mistral e Llama 3.2 Vision), avaliados com base em métricas de latência, qualidade textual (BERTScore e ROUGE-L) e testes práticos em cenários reais. Os resultados demonstraram que o modelo Qwen 2.5 apresentou o melhor equilíbrio entre precisão descritiva e desempenho em tempo real. O protótipo desenvolvido evidencia o potencial da IA para ampliar a inclusão digital, destacando-se como uma solução inovadora para a melhoria da qualidade de vida de pessoas com deficiência visual.

	\vspace{\onelineskip}
	
	\textbf{Palavras-chave}: Acessibilidade Digital, Tecnologia Assistiva, Modelos de Linguagem Multimodais, Visão Computacional, Inclusão Social.
	
\end{resumo}